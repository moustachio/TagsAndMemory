\section{Introduction}
\label{sec_intro}

In social tagging systems, users assign freeform textual labels to digital content (music, photos, web bookmarks, etc.). These individual tagging decisions are aggregated into a folksonomy \cite{VanderWal2007}, a ``bottom-up'' classificatory structure developed with little or no top-down guidance or constraints. There are a variety of reasons for which users tag content, but it is overwhelming assumed that tagging for future retrieval -- assigning a tag to an item to facilitate re-finding it at a later time -- is users' principal motivator. But is this a valid assumption?

Collaborative tagging systems are often designed, at least in part, as resource management platforms that expressly facilitate the use of tags as retrieval aids, but the freeform, and often social, nature of tagging opens up many other possible reasons for which a user might tag a resource. There is much non-controversial evidence for such alternative tagging motivations, (sharing resources with other users, evaluation, etc.) but the problem with the retrieval aid assumption runs deeper than there simply existing possible alternatives. There is, in fact, almost no behavioral evidence that tags are ever actually used as retrieval aids. While there is much data available user tagging habits (i.e. which terms are applied to which resources, and when), to our knowledge there is no published research providing behavioral evidence of whether or not tags, once applied to items, actually facilitate subsequent retrieval. This is an issue largely driven by a lack of data: While a web service can in principle track users' interaction with tags (for instance, if users use tags as search terms to find tagged content), there are no available datasets containing such information, nor can it be crawled externally by researchers.

The problem is not intractable, however. While measuring how existing tags are utilized remains beyond our reach, an alternative approach is to examine how patterns of user interaction with tagged versus untagged content vary. In other words, if tags do serve as retrieval aids, we should expect users to be more likely to interact with resources (e.g. visit bookmarked pages, listen to songs, view photos, etc.) upon the application of a tag.

In the current paper we test this hypothesis using a large-scale dataset from the social music website Last.fm using, consisting of complete listening and tagging histories from more than 100,000 users. From this we extract user-artist listening time series (each representing the frequency of listening over time to a particular artist by a particular user), and compare those time series in which the user has tagged the artist, and those that are untagged. Specifically, we address the following two questions:
\begin{itemize}
	\item RQ1: Does comparison of tagged versus untagged time series provide evidence that tagging an artist increases probability of listening to that artist in the future?
	\item RQ2: Do certain tags prove to be particularly associated with increases in future listening, and if so, can we identify attributes of such ``retrieval-targeted'' tags as opposed to others?
\end{itemize}

We describe the various analytic methods we bring to bear on these questions in Section~\ref{sec_analyses}, but first present related work (Section~\ref{sec_related}) and details of our dataset (Section~\ref{sec_dataset}). We close in Section~\ref{sec_conclusion} with synthesis and interpretation of our results, as well as a plan for future work.