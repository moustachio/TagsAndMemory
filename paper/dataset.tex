\section{Dataset}
\label{sec_dataset}
Last.fm incorporates two specific features of interest to us here. First, it implements a collaborative tagging system (a ``broad'' folksonomy, following Vander Wal's \cite{VanderWal2005} terminology, meaning that multiple users tag the same, publicly available content) in which users can label artist, albums, and songs. Second, the service tracks users' listening habits both on the website itself and on media players (e.g. iTunes) via a software plugin. This tracking process is known as ``scrobbling'', and each timestamped instance of a user listening to a particular song is termed a ``scrobble''.

Here we utilize an expanded version of a dataset described in earlier work \cite{Lorince2013,Lorince2014} that includes the full tagging histories of approximately 1.9 million Last.fm users, and full listening histories from a subset of those users (approximately 100,000) for a 90-month time window (July 2005 - December 2012, inclusive). Data were collected via a combination of the Last.fm API and direct scraping of publicly available user profile pages. For further details of the crawling process, see \cite{Lorince2013,Lorince2014}.

For our current purposes, we consider only those users for which we have both tagging and listening histories. For each user, we extract one time series for each unique artist listened to by that user. Each user-artist listening time series consists of a given users's monthly listening frequency to a particular artist for each month in our data collection period, represented as a 90-element vector.

User tagging histories are only available at monthly time resolution, so we also downsample scrobble data (which is recorded to second precision) to monthly playcounts as well. Furthermore, we perform all analyses here at the level of artists, rather than individual songs. Thus every song scrobbled is treated as a listen to the corresponding artist, and all annotations (which can be applied to songs, albums, or artists) are treated as annotations of the corresponding artist. Our choice to perform all analyses at the level of artists, rather than individual songs, is based on the facts that (a) listening and tagging data for any particular song tends to be very sparse, and (b) the number of time series resulting from considering each unique song listened to by each user would be prohibitively large.

The over 2 billion individual scrobbles in our dataset define a total of XXX user-artist listening time series. In XXX of these cases, the user has assigned at least one tag to the artist (or to a song or album by that artist) within the collection period (we refer to these as tagged time series), while in the remaining cases (~89 million) the user has never tagged the artist. We summarize these high level dataset statistics in Table~\ref{tab:data_summary}. Comparison of these these tagged and untagged listening time series is the heart of the analyses presented in the next section.

\begin{table}[h]
\begin{center}
\begin{tabular}{l|r}
\toprule
Total users & 104,829 \\
Total scrobbles & 2,089,473,214 \\
Unique artists listened & 4,444,119 \\
Unique artists tagged & 1,049,263 \\
\midrule
Total user-artist listening time series & XX,XXX,XXX \\
Total tagged time series & X,XXX,XXX \\
Total untagged time series & 88,944,512 \\
\bottomrule
\end{tabular}
\end{center}
\caption{Dataset summary}
\label{tab:data_summary}
\end{table}

