
%%%%%%%%%%%%%%%%%%%%%%% file typeinst.tex %%%%%%%%%%%%%%%%%%%%%%%%%
%
% This is the LaTeX source for the instructions to authors using
% the LaTeX document class 'llncs.cls' for contributions to
% the Lecture Notes in Computer Sciences series.
% http://www.springer.com/lncs       Springer Heidelberg 2006/05/04
%
% It may be used as a template for your own input - copy it
% to a new file with a new name and use it as the basis
% for your article.
%
% NB: the document class 'llncs' has its own and detailed documentation, see
% ftp://ftp.springer.de/data/pubftp/pub/tex/latex/llncs/latex2e/llncsdoc.pdf
%
%%%%%%%%%%%%%%%%%%%%%%%%%%%%%%%%%%%%%%%%%%%%%%%%%%%%%%%%%%%%%%%%%%%


\documentclass[runningheads,a4paper]{llncs}

\usepackage{amssymb}
%\usepackage{amsthm}
\setcounter{tocdepth}{3}
\usepackage{graphicx}
\usepackage{booktabs}

\usepackage[font={small}]{caption, subfig}
\usepackage{url}
\urldef{\mails}\path|{jlorince,pmtodd}@indiana.edu| 
\newcommand{\keywords}[1]{\par\addvspace\baselineskip
\noindent\keywordname\enspace\ignorespaces#1}

%\pdfpagesattr{/CropBox [94 112 523 778]}
  \def\baselinestretch{0.98}
  \addtolength{\textheight}{0.7in}
\begin{document}

\mainmatter  % start of an individual contribution

% first the title is needed
\title{Analysis of music tagging and listening patterns: Do tags really function as retrieval aids?}


% a short form should be given in case it is too long for the running head
\titlerunning{Do tags really function as retrieval aids?}

% the name(s) of the author(s) follow(s) next
%
% NB: Chinese authors should write their first names(s) in front of
% their surnames. This ensures that the names appear correctly in
% the running heads and the author index.
%
\author{Jared Lorince\inst{1}%
\and Kenneth Joseph\inst{2} \and Peter M. Todd\inst{1}}
%
\authorrunning{Lorince, Joseph, \& Todd}
% (feature abused for this document to repeat the title also on left hand pages)

% the affiliations are given next; don't give your e-mail address
% unless you accept that it will be published
\institute{Cognitive Science Program\\
%Department of Psychological \& Brain Sciences\\
Indiana University, Bloomington, Indiana, USA\\
\mails \\
\and 
Computation, Organization and Society Program \\
%School of Computer Science\\
Carnegie Mellon University, Pittsburgh, PA, USA\\
\email{kjoseph@cs.cmu.edu}}

%
% NB: a more complex sample for affiliations and the mapping to the
% corresponding authors can be found in the file "llncs.dem"
% (search for the string "\mainmatter" where a contribution starts).
% "llncs.dem" accompanies the document class "llncs.cls".
%

\toctitle{Analysis of music tagging and listening patterns: Do tags really function as retrieval aids?}
\tocauthor{Lorince, Joseph, \& Todd}
\maketitle

\begin{abstract}
In collaborative tagging systems, it is generally assumed that users assign tags to facilitate retrieval of content at a later time. There is, however, little behavioral evidence that tags actually serve this purpose. Using a large-scale dataset from the social music website Last.fm, we explore here how patterns of music tagging and subsequent listening interact in an effort to determine if there exist measurable signals of tags functioning as retrieval aids. Specifically, we describe several methods for testing if the assignment of a tag tends to lead to an increase in listening behavior. Results indicate \ldots
\keywords{Collaborative tagging, Folksonomy, Music listening, Memory cues, Retrieval aids, Personal information management}
\end{abstract}
\setcounter{footnote}{0}

\section{Introduction}
\label{sec_intro}

\section{Related work}
\label{sec_related}
\section{Dataset}
\label{sec_dataset}
Last.fm incorporates two specific features of interest to us here. First, it implements a collaborative tagging system (a ``broad'' folksonomy, following Vander Wal's \cite{VanderWal2005} terminology, meaning that multiple users tag the same, publicly available content) in which users can label artist, albums, and songs. Second, the service tracks users' listening habits both on the website itself and on media players (e.g. iTunes) via a software plugin. This tracking process is known as ``scrobbling'', and each timestamped instance of a user listening to a particular song is termed a ``scrobble''.

Here we utilize an expanded version of a dataset described in earlier work \cite{Lorince2013,Lorince2014} that includes the full tagging histories of approximately 1.9 million Last.fm users, and full listening histories from a subset of those users (approximately 100,000) for a 90-month time window (July 2005 - December 2012, inclusive). Data were collected via a combination of the Last.fm API and direct scraping of publicly available user profile pages. For further details of the crawling process, see \cite{Lorince2013,Lorince2014}.

For our current purposes, we consider only those users for which we have both tagging and listening histories. For each user, we extract one time series for each unique artist listened to by that user. Each user-artist listening time series consists of a given users's monthly listening frequency to a particular artist for each month in our data collection period, represented as a 90-element vector.

User tagging histories are only available at monthly time resolution, so we also downsample scrobble data (which is recorded to second precision) to monthly playcounts as well. Furthermore, we perform all analyses here at the level of artists, rather than individual songs. Thus every song scrobbled is treated as a listen to the corresponding artist, and all annotations (which can be applied to songs, albums, or artists) are treated as annotations of the corresponding artist. Our choice to perform all analyses at the level of artists, rather than individual songs, is based on the facts that (a) listening and tagging data for any particular song tends to be very sparse, and (b) the number of time series resulting from considering each unique song listened to by each user would be prohibitively large.

The over 2 billion individual scrobbles in our dataset define a total of XXX user-artist listening time series. In XXX of these cases, the user has assigned at least one tag to the artist (or to a song or album by that artist) within the collection period (we refer to these as tagged time series), while in the remaining cases (~89 million) the user has never tagged the artist. We summarize these high level dataset statistics in Table~\ref{tab:data_summary}. Comparison of these these tagged and untagged listening time series is the heart of the analyses presented in the next section.

\begin{table}[h]
\begin{center}
\begin{tabular}{l|r}
\toprule
Total users & 104,829 \\
Total scrobbles & 2,089,473,214 \\
Unique artists listened & 4,444,119 \\
Unique artists tagged & 1,049,263 \\
\midrule
Total user-artist listening time series & XX,XXX,XXX \\
Total tagged time series & X,XXX,XXX \\
Total untagged time series & 88,944,512 \\
\bottomrule
\end{tabular}
\end{center}
\caption{Dataset summary}
\label{tab:data_summary}
\end{table}


\section{Analyses \& Results}
\label{sec_analyses}

\subsection{Comparison of tagged and untagged time series}
Our principal research question is whether listening patterns for tagged content are consistent with the expectation that tags serve as memory cues. If this were to be the case, we would expect to see increased listening rates for musical artists once a tag is applied, under the assumption that a tag facilitates retrieval and increaes the chances of a user listening to a tagged artist. 

Unfortunately, several factors combine to make such an analysis difficult.  First and foremost, the desired counterfactual of the untagged ``version'' of a particular tagged series does not, of course, exist. We thus must utilize untagged time series in a way that allows them to approximate what a true counterfactual might look like.  In searching for such samples, a second difficulty that arises is that listening rates for tagged time series are much greater than for untagged time series (the average number of total listens across time series is 16.9  when untagged and 98.9 when tagged). While suggestive of the importance of tagging, this unbalance also suggests that controls must be instilled in both sample selection and statistical analysis to account for previous listening behavior prior to tagging. Finally, the actual point in time at which tags are expected to increase listening behavior for any given user is unknown, as it is theoretically possible that tagging may affect listening behavior as much three months after the tag has been placed as it does in the immediately following month.  Thus, we must formulate our analysis in such a fashion as to account for this possibility.  
%to understand how it changes behavior

%meticulously
To alleviate issues with the non-existence of a true counterfactual, we subselect from both the tagged and untagged series using the following formal procedure. We first temporally align the tagged and untagged time series. Tagged time series are aligned so that they are centered on the month in which they were tagged.  While there is no analogue to this point in the untagged data, we can partially resolve the issue by noting that tagging is disproportionately likely (approximately 30\%, compared to 1.1\% if the tagging month were random) to occur in a user's \emph{peak}\footnote{The month in which they listen the most times} listening month for a given artist. This provides a basis for aligning the tagged and untagged time series, by selecting only those tagged time series where the tag was applied in the month of peak listening, and then collecting a sample of untagged time series also aligned at the peak of listening. After aligning all tagged and untagged samples in this fashion, we further limited our analysis to a 13 month period extending from 6 months prior to the peak month to 6 months after the peak. This data allows us to consider a variety of ways in which listening prior to the tag may affect future behavior, as well as the case where future behavior is affected well after the tag has been applied. Finally, we further constrain our sampling to time series with:
\begin{itemize}
\item more than 25 total listens; 
\item a peak in listening at least 6 months from the edges of our data collection period (i.e. ensuring that the period from 6 months before to 6 months after the peak does not extend beyond the limits of our data range); and
\item at least one listen 6 months prior to and after the peak (i.e. if the peak occurs in July 2008, there should be at least one listen in January 2007, and one listening in January 2009).
\end{itemize}

Constraining our time series in this manner, we are left with a total 206,140 tagged time series, and randomly sampled from the 4.1M matching untagged time series an equal number meeting the same three criteria.  All results below have been verified with multiple random samplings of the untagged data.

In Figure~\ref{fig:taggedVsUntagged} we plot mean playcounts, with 95\% normal confidence intervals, for tagged and untagged time series for each month in our analysis window. All values are normalized by the peak, and thus values at the peak month for both the tagged and untagged lines are unity. We observe that there does exist an increase in the mean normalized listening rate in the months after the peak as opposed to those prior for both tagged and untagged time series. Additionally, and more importantly, we also see a small but reliable effect wherein tagged time series show proportionally higher mean normalized listening rates after the peak month (in which the the tag was applied) as compared to untagged time series. This is suggestive of an increase in listening as a result of tagging.

While Figure~\ref{fig:taggedVsUntagged} indicates a slight difference in the normalized means of the tagged and untagged data that supports our hypothesis, there are two important caveats to the data in the plot. First, because listening distributions are heavily skewed for any given month, the mean is not necessarily representative of the distribution. Though qualitative plots will be similar, further statistical analysis should use either a transformed version of the listening counts or an appropriate count-based methodology. Second, while normalization controls for differences in listening counts to some extent, we would prefer a method to explicitly account for listening behavior prior to and including the peak month on future behavior.

To more robustly test our hypothesis, we thus utilize a regression model relating post- and pre-peak listening behavior. As we are unsure at what point post-tagging the effect of a tag may be strongest, we consider as an outcome variable the logarithm of the sum of all listens in the six months after a tag has been applied.\footnote{Qualitative, our results hold in each individual month. Also of note is our choice of using the log of the dependent variable rather than a count-based regression model. The model used here appeared to fit better than the count-based methods we attempted.} Our independent variables are an indicator of whether or not the time series has been tagged, as well as the logarithm of the sum of listens for the peak month and the six previous months.  Finally, due to the volume of data we are dealing with, it was unreasonable make the assumption of linear dependence of the dependent variable on the independent variables. We therefore opted for a Generalized Additive Model (GAM, \cite{hastie1990generalized}), for which we utilized the R package mgcv \cite{wood2001mgcv}.

  \begin{figure}
    \subfloat[Mean normalized playcount by month. \label{fig:taggedVsUntagged}]{%
      \includegraphics[width=0.5\textwidth]{taggedVUntaggedSimple.png}
    }
    \hfill
    \subfloat[Regression results, with 95\% confidence interval. \label{fig:regression}]{%
      \includegraphics[width=0.45\textwidth]{taggedVUntaggedRegression.png}
    }
    \caption{Comparison of tagged and untagged listening time series}
    \label{fig:regressionFigs}
  \end{figure}

The regression model, which explained approximately 30\% of the variance in the data (adj. R-sq.), indicated that smoothed parameters for all previous months had a signficant effect on post-peak listening behavior.  As we cannot show the form of this effect for all model variables, Figure~\ref{fig:regression} instead displays a similar model which considers only the effect of listening in the peak month on post-peak listening. As this plot suggests and the full model confirms, we can conclude that, controlling for all previous listening behavior, a tag increases the logarithm of post-peak listens by .147 [.144,.150]. Thisindicates that the effect of a tag is associated with around 1.15 more listens, on average, than if it were not to have been applied.   

\subsection{Tag analysis}
To examine if and how different tags are associated with increased future listening, we ran a regression analysis similar to that described above, except for three changes. First, we eliminated the constraint that an annotation must occur in the pea month of a time series, as there is no meaninguful comparison to be made with untagged data in this analysis. Second, due to the data-hungry nature of the GAM, we chose to only control for listening in the peak month. This decision limited the computational difficulties associated with estimating the model and did not appear to affect model fit substantially in subsamples of the data. Third, instead of a single tagged/untagged indicator, we included binary (present / not present) regressors for all unique tags that had at least 25 occurrences in our subsample.  This expanded sample consisted of XXX,XXX tagged time series, with X,XXX unique tags. After running the model, which explains $\sim$XX\% of the variance in the data (adj. R-sq.), XXX unique tags  proved to be statistically significant predictors at $p <.001$. While we only have sufficient evidence to make claims about these XXX tags, qualitative examination of which tags are relatively strong predictors in the model proves informative.

The most telling observation is that commonly-used genre tags (e.g. ``pop'', ``jazz'', and ``hip-hop'') -- which are the most common tags overall in our full dataset -- tend to be weak, negative predictors of future listening. In contrast, relatively strong predictors (both positive and negative) appear to be comparatively obscure, possibly idiosyncratic tags (``arguman-loved tracks'', ``mymusic'', ``leapsandbounds cdcollection'').\footnote{For a full listing of the regression coefficients across all tags in the model, see \url{https://dl.dropboxusercontent.com/u/625604/papers/lorince.joseph.todd.2015.sbp.supplemental/regression_coefficients.txt}} To examine this trend quantitatively, we plot in Figure~\ref{fig:coefVsPopularity} global tag popularity (i.e. the total number of uses of a tag in our full dataset, which consists of $\sim 50$ million annotations) as a function of the tag's coefficient in the regression model. The red bands marked the upper and lower limits of a bootstrapped 95\% confidence interval on the popularity of the remaining XXX tags that were \emph{not} significant in the regression model. The result is a clear trend which suggests that the most popular tags are significant, weakly negative predictors of future listening, while both positive and negatively strong predictors tend to be relatively unpopular. Tags which were not significant in the model tend to be of moderate to high popularity

  \begin{figure}
	\centering
      \includegraphics[width=0.7\textwidth]{scatterplotTagRegression.png}
    \caption{Logarithm of tags' global popularity as a function of regression coefficient.}
    \label{fig:coefVsPopularity}
  \end{figure}

These data suggest that, at least for the small number of tags about which we can make statistically meaningful claims, those that are globally popular and well-known have relatively little effect on future listening, and are generally associated with small decreases in post-taggging listening rates. The tags that seem to ``matter'' (i.e. those that are relatively strong predictors of whether or not a user will listen to an artist after tagging it) are generally much less popular.
\section{Conclusion}
\label{sec_conclusion}
In this paper we set out to test the oft-cited assumption that tags  seve as retrieval aids for indivduals in collaborative tagging systems via a novel methoddology, testing for evidence that tagging an artist increases a user's future listening to that artist. Results suggest that tagging an artist does lead to an increase in listening, but that this increase is, on average, quite small (amounting to only 1 or 2 listens over a 6 month period). Given the various possible motivations for tagging, however, we expect only some tags to serve as retrieval cues, and thus tested the relative predictiveness of future listening for different tags. This analysis revealed systematic differences in how predictive the presence or absence of different tags was for future listening as a function of tag popularity. Specifically, we found that the most popular tags tend to have a small or non-significant effect on future listening, while less popular tags appear to be those that ``matter'', both as positive and negative predictors of future listening.

Based on such a small sample, we are at this point tentative to make strong claims about what specifically differntiates those unpopular tags that are strong negative versus strong positive predictors. The evidence is, nevertheless, suggestive of relatively uncommon (and likely, in many cases, to be idiosyncratic) tags being those most predictive of future listening behavior. This raises the intriguing possibility that the descriptive, popular tags that are arguably most useful to the community at large (i.e. genre labels and related tags), are not particularly associated with increases in listening, and thus are likely not functioning as memory cues. 

This suggests that, while on average tagging an artist  has a small positive effect on future listening, the most common tagging activities are \emph{not} are not strong predictors of future retrieval listneing. We cannot be sure which of the many other possible tagging motivations are at play here, nor can we know if and when a tag is applied with the intention of being used for retrieval, while ultimately not being used for this purpose. That said, these results do suggest that descriptive, relatively well-known genre classifiers do not show evidence of use as retrieval aids, but are nonetheless the most commonly applied tags. This may indicate that the primary motivation for tagging on Last.fm is not for personal information management (tagging a resource for one's own retrieval), but rather is socially-oriented, resulting in tags that are useful for the community at larger.

This leads to the interesting possibility that a folksonomy can generate the useful, crowdsourced classification of content that proponents of collaborative tagging extol, but that this process is not strongly driven by the self-directed, retrieval-oritented tagging that is typically assumed in such systems.

While our results provide clues as to whether tags really function as retrieval aids, this remains early stage work addressing a hitherto unstudied research question. There is certainly room to refine and build upon the methods we present here for testing if and when tagging increases listening rates. In particular, our analysis at the level of artists (rather than the individual resources tagged) may be problematic, and we hope to eventually develop models that operate direclty at the level of the content tagged (though data sparsity issues will make this a challenge). It will also be critical to expand on methods for understandings which tags serve as memory cues and under what circumstances. It is clearly the case that not \emph{all} tags function as memory cues, so more robustly identifying which do is fruitful direction for future work. Incorporating research on human memory from the cognitive sciences can also inform hypotheses and analytic approaches to these questions, something we are actively pursuing in ongoing research. A final limitation is that we are exploring tagging in a particular collaborative tagging system, which operates in the possibly idiosyncratic domain of music. Tagging habits may vary systematically in different cotent domains, but until usable data becomes avaialble, we can only speculate as to exactly how.

In closing, to address the titular question of whether or not tags function as retrieval aids, the best answer would appear to be ``sometimes''. While there is much work to be done on when and why particular tags serve this function and others do not, it is clear that the overarching retrieval assumption is far from valid: Tags certainly do not always function as memory cues, and our results suggest that retrieval may actually be among one of the least common of tagging motivations. 



\bibliographystyle{splncs}
\bibliography{references}

\end{document}
