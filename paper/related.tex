\section{Background}
\label{sec_related}

%\subsection{The formal study of folksonomies}
Collaborative tagging has been considered one of the core technologies of ``Web 2.0'', and has been implemented for resources as diverse as web bookmarks (Delicious), photos (Flickr), books (LibraryThing), academic papers (Mendeley), and more. Vander Wal \cite{VanderWal2007} coined the term ``folksonomy'' to describe the emergent semantic structure defined by the aggregation of many individual users' tagging decisions in such a system. These folksonomies have since become the target of much academic research. One of the earliest analyses of a collaborative tagging system is Golder and Huberman's \cite{Golder2006} work on the evolution of tagging on Delicious.com.  Around the same time, Hotho and colleagues \cite{Hotho2006a} presented a formal definition of a folksonomy: $\mathbb{F} := (U,T,R,Y)$ \footnote{This is a slight simplification. For details, see \cite{Hotho2006a}}. The variables $U$, $T$, and $R$ represent, respectively, the sets of users, tags, and resources in a tagging system, while $Y$ is a ternary relation between them ($Y \subseteq U \times T \times R$). The ``personomy'' of a particular user (i.e. the set of annotations generated by an individual),  $\mathbb{P} := (T_{u},R_{u},Y_{u})$, can be similarly defined.

Since 2006, an extensive literature on \emph{how} people tag has also developed, covering topics like tagging expertise \cite{Yeung2011},	mathematical \cite{Cattuto2007a} and multi-agent \cite{Lorince2013} models of tagging choices, consensus in collaborative tagging \cite{Halpin2007,Robu2009}, and much more. Our understanding of the dynamics of tagging behavior has greatly expanded, but understanding exactly \emph{why} people tag, on the other hand, has proven more elusive.

%\subsection{Why do people tag?}
It is typically assumed that tags serve as retrieval aids, allowing users to re-find content to which they have applied a given tag (e.g. a user could click on or search for the tag ``rock'' to retrieve the songs she has previously tagged with that term). This assumption is baked into Vander Wal's original defintion of a folksonomy, which he contends ``is the result of personal free tagging of information and objects (anything with a URL) \emph{for one's own retrieval}'' \cite[emphasis added]{VanderWal2007}. This perspective is echoed in many studies of tagging patterns \cite{Glushko2008,Halpin2007,Golder2006}.

But while retrieval is the most commonly assumed motivation for tagging, other reasons certainly exist, and various researchers have developed taxonomies of tagging motivation. Among proposed motivational factors in tagging are personal information management (including but not limited to tagging for future retrieval), resource sharing, opinion expression, performance, and activism \cite{Heckner2009,Zollers2007,Ames2007}, among others. See \cite{Gupta2010} for a review.

While the development of motivational theories in tagging is useful, there is almost no work actually grounding them in behavioral observations. The vast majority of existing work either makes inferences about motivation based on design features of a website (e.g. social motivations in tagging require that one's tags be visible to other users, \cite{Marlow2006}), employs semantic analysis and categorization of tags (e.g. the tags ``to read'', ``classical'', and ``love'' can all be inferred to have different uses, \cite{Zollers2007,Sen2006}), or directly asks users why they tag using survey methods \cite{Ames2007,Nov2008}. The results of such approaches are useful contributions to the field, but few have resulted in testable behavioral hypotheses that can confirm or refute their validity.

One notable exception is work by K\"{o}rner and colleagues \cite{Korner2010,Korner2010a,Zubiaga2011}. They argue that taggers can be classified on a motivational spectrum from categorizers (who use a constrained vocabulary suitable to future browsing of their own tagged resources) to describers (who use a large, varied vocabulary to facilitate future keyword-based search using their own tags), and have developed and tested quantifiable signals of these different motivations. The main deficiency of this approach, however, is that their hypotheses are based fully on attributes of user tag vocabularies; they present no way to test whether or not describers actually use tags, once applied, for keyword-based search and that categorizers use them for browsing.

Again, the problem of lack of verification arises because data on how users actually \emph{use} existing tags is simply not available to researchers through any tagging system APIs (or through other methods) that we are aware of. Thus the existing work on tagging motivation is limited to inferring \emph{why} people tag from \emph{how} they tag, rather than from how they \emph{use} their tags. In presenting our novel methods, we are aware that they still represent an inferential approach. Our approach is distinct from those described here, however, in that we test a concrete hypothesis about how tagging should affect a behavior on which we \emph{do} have data (interaction with tagged content, in our case music listening).
%\subsection{Insights from cognitive science}