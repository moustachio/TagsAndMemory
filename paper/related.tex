\section{Background}
\label{sec_related}

\subsection{The formal study of folksonomies}
Collaborative tagging has been considered on of the core technlogies of "Web 2.0", and has been implemented for resources as diverse as web Bookmarks (Delicious), photos (Flickr), books (LibraryThing), academic Papers (Mendley), and more. Thomas Vander Wal \cite{VanderWal2007} first coined the term ``folksonomy'' to describe the emergent semantic structure defined by the aggregation of many individual users' tagging decisions in such a system, which of since become the target of much academic research. One of the earliest well-known and involved analyses of a collaborative tagging system is Golder and Huberman's \cite{Golder2006} analysis of the evolution of tagging on Delicious.com, and in the same year Hotho and colleagues \cite{Hotho2006a} presented a formal defintion of a folksonomy: $\mathbb{F} := (U,T,R,Y)$ \footnote{This is a slight simplication. For details, See \cite{Hotho2006a}}. $U$, $T$, and $R$ represent, respectively, the sets of users, tags, and resources in a tagging system, while $Y$ is  aternary relation between them ($Y \subseteq U \times T \times R$). The ``personomy'' of a particular user (i.e. the set of resources tagged by an individual),  $\mathbb{P} := (T_{u},R_{u},Y_{u})$, can be similarly defined.

Since 2006, an extensive literature on \emph{how} people tag has been developed, covering topics like tagging expertise \cite{Yeung2009,Yeung2011},	mathematical \cite{Cattuto2007} and multi-agent \cite{Lorince2013} models of tagging choices, consensus in collaborative tagging \cite{Robu2009}, and much more. Our understanding of the dynamics of tagging behavior has greatly expanded, but understadning exactly \emph{why} people tag, on the other hand, has proven more elusive.

\subsection{Why do people tag?}

\subsection{Insights from cognitive science}