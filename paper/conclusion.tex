\section{Conclusion}
\label{sec_conclusion}

In this paper we set out to test the oft-cited assumption that tags serve as retrieval aids for individuals using collaborative tagging systems.  We did so via a novel methodology, testing for evidence that tagging an artist increases a user's future listening to that artist in comparison to a carefully selected set of untagged time-series. Results suggest that tagging an artist does lead to an increase in listening, but that this increase is, on average, quite small (amounting to only 1 or 2 listens over a 6 month period). Given the various possible motivations for tagging, however, we expect only some tags to serve as retrieval cues, and thus tested the relative predictiveness of future listening for different tags. This analysis revealed systematic differences in how predictive the presence or absence of different tags was for future listening as a function of tag popularity. The data suggest that, at least for the small number of tags about which we can make statistically meaningful claims, those that are globally popular and well-known have relatively little effect on future listening, and are generally associated with very small increases in post-taggging listening rates. The tags that seem to ``matter'' (i.e. those that are relatively strong predictors of whether or not a user will listen to an artist after tagging it) are generally much less popular. Even these stronger predictors, however, lead to relatively slight increases in listening. The strongest predictors are associated with only about 4.5 listens over a six month period, on average. %Specifically, we found that the most popular tags tend to have a small or non-significant effect on future listening, while less popular tags appear to be those that ``matter'', both as positive and negative predictors.

Because we only had a small sample of statistically influential tags to analyze, we are at this point tentative to make strong claims about which specific factors contribute to tags being better or worse predictors of increased listening, or under what circumstances they predict \emph{decreases} in listening.\footnote{There were few enough of them in the current analysis as to suggest they may be largely due to statistical noise.} The evidence is, nevertheless, suggestive that relatively uncommon (and in many cases idiosyncratic) tags are most predictive of future listening behavior. The intriguing flipside is that the descriptive, popular tags that are arguably most useful to the community at large (i.e. genre labels and related tags) are not particularly associated with increases in listening, and thus are likely not functioning as memory cues.

This suggests that, while on average tagging an artist  has a small positive effect on future listening, the most common tagging activities are \emph{not} strong predictors of future retrieval. We cannot be sure which of the many other possible tagging motivations are at play here, nor can we tell at this point if and when a tag is applied with the intention of being used for retrieval, while ultimately not being used for this purpose. That said, our results may indicate that the primary motivation for tagging on Last.fm is not for personal information management (tagging a resource for one's own retrieval), but rather is socially oriented, resulting in tags that are useful for the community at large. This leads to the interesting possibility that a folksonomy can generate the useful, crowdsourced classification of content that proponents of collaborative tagging extol, even if this process is not strongly driven by the self-directed, retrieval-oriented tagging that is typically assumed in such systems.

While our results provide clues as to whether tags really function as retrieval aids, this remains early work addressing a hitherto unstudied research question. There is certainly room to refine and build upon the methods we present here for testing if and when tagging increases listening rates. In particular, our analysis at the level of artists (rather than the individual resources tagged) may be problematic, and we hope to eventually develop models that operate directly at the level of the content tagged (though data sparsity issues will make this a challenge). There are also many factors we have not controlled for here that could be incorporated into future models, such as exogenous influences on listening (e.g. when an artist releases a new album). It will be critical to expand on methods for understanding which tags serve as memory cues and under what circumstances. It is clearly the case that not \emph{all} tags function as memory cues, so more robustly identifying which tags do serve as memory retrieval items is a fruitful direction for future work. Incorporating research on human memory from the cognitive sciences can also further inform hypotheses and analytic approaches to these questions, something we are actively pursuing in ongoing research. A final limitation is that we are exploring tagging in a particular collaborative tagging system, which operates in the possibly idiosyncratic domain of music. Tagging habits may vary systematically in different content domains, but until usable data becomes available, we can only speculate as to exactly how. \todo[inline]{Note the new sentence (``There are also many...'') to help address reviewer 3's comments}

In closing, to address the titular question of whether or not tags function as retrieval aids, the best answer would appear to be ``sometimes, but usually not''. While there is much work to be done on when and why particular tags serve this function and others do not, it is clear that the overarching retrieval assumption is far from universally valid: Tags certainly do not always function as memory cues, and our results suggest that retrieval may actually be an uncommon tagging motivation.

