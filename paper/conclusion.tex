\section{Conclusion}
\label{sec_conclusion}

\todo[inline]{Need to think through this section a bit more, once results are a bit more finalized, but the following chunk definitely should be in there (though certainly with some edits).}
Based on such a small sample, we are at this point tentative to make strong claims about what specifically differntiates those unpopular tags that are strong negative versus strong positive predictors of future listening. The evidence is suggestive of, nevertheless, relatively uncommon (and likely, in many cases, to be idiosyncratic) tags being those most predictive of future listening behavior. This raises the intriguing possibility that the descriptive, popular tags that are arguably most useful to the community at large (i.e. genre labels and related tags), are not particularly strong predictors of future listening, and thus are likely not functioning as memory cues. 

This suggest that, while on average tagging an artist  has a small positive effect on future listening, the most common tagging activities are \emph{not} are not strong predictors of future retrieval listneing. We cannot be sure which of the many other possible tagging motivations are at play here, nor can we know if and when a tag is applied with the intention of being used for retrieval, while ultimately not being used for this purpose. That said, these results do suggest that descriptive, relatively well-known genre classifiers do not show evidence of use as retrieval aids, but are nonetheless the most commonly applied tags. This may indicate that the primary motivation for tagging on Last.fm is not for personal information management (tagging a resource for one's own retrieval), but rather is socially-oriented, resulting in tags that are useful for the community at larger.

This leads to the interesting possibility that a folksonomy can generate the useful, crowdsourced classification of content that proponents of collaborative tagging extol, but that this process is not strongly driven by the self-directed, retrieval-oritented tagging that is typically assumed in such systems.

\todo[inline]{Need to talk about limitations, both of methods, and of course that we're talking just about one system...could be different on, say, Delicious, but we don't yet have to test it}