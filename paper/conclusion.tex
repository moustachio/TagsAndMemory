\section{Conclusion}
\label{sec_conclusion}
In this paper we set out to test the oft-cited assumption that tags  seve as retrieval aids for indivduals in collaborative tagging systems via a novel methoddology, testing for evidence that tagging an artist increases a user's future listening to that artist. Results suggest that tagging an artist does lead to an increase in listening, but that this increase is, on average, quite small (amounting to only 1 or 2 listens over a 6 month period). Given the various possible motivations for tagging, however, we expect only some tags to serve as retrieval cues, and thus tested the relative predictiveness of future listening for different tags. This analysis revealed systematic differences in how predictive the presence or absence of different tags was for future listening as a function of tag popularity. Specifically, we found that the most popular tags tend to have a small or non-significant effect on future listening, while less popular tags appear to be those that ``matter'', both as positive and negative predictors of future listening.

Based on such a small sample, we are at this point tentative to make strong claims about what specifically differntiates those unpopular tags that are strong negative versus strong positive predictors. The evidence is, nevertheless, suggestive of relatively uncommon (and likely, in many cases, to be idiosyncratic) tags being those most predictive of future listening behavior. This raises the intriguing possibility that the descriptive, popular tags that are arguably most useful to the community at large (i.e. genre labels and related tags), are not particularly associated with increases in listening, and thus are likely not functioning as memory cues. 

This suggests that, while on average tagging an artist  has a small positive effect on future listening, the most common tagging activities are \emph{not} are not strong predictors of future retrieval listneing. We cannot be sure which of the many other possible tagging motivations are at play here, nor can we know if and when a tag is applied with the intention of being used for retrieval, while ultimately not being used for this purpose. That said, these results do suggest that descriptive, relatively well-known genre classifiers do not show evidence of use as retrieval aids, but are nonetheless the most commonly applied tags. This may indicate that the primary motivation for tagging on Last.fm is not for personal information management (tagging a resource for one's own retrieval), but rather is socially-oriented, resulting in tags that are useful for the community at larger.

This leads to the interesting possibility that a folksonomy can generate the useful, crowdsourced classification of content that proponents of collaborative tagging extol, but that this process is not strongly driven by the self-directed, retrieval-oritented tagging that is typically assumed in such systems.

While our results provide clues as to whether tags really function as retrieval aids, this remains early stage work addressing a hitherto unstudied research question. There is certainly room to refine and build upon the methods we present here for testing if and when tagging increases listening rates. In particular, our analysis at the level of artists (rather than the individual resources tagged) may be problematic, and we hope to eventually develop models that operate direclty at the level of the content tagged (though data sparsity issues will make this a challenge). It will also be critical to expand on methods for understandings which tags serve as memory cues and under what circumstances. It is clearly the case that not \emph{all} tags function as memory cues, so more robustly identifying which do is fruitful direction for future work. Incorporating research on human memory from the cognitive sciences can also inform hypotheses and analytic approaches to these questions, something we are actively pursuing in ongoing research. A final limitation is that we are exploring tagging in a particular collaborative tagging system, which operates in the possibly idiosyncratic domain of music. Tagging habits may vary systematically in different cotent domains, but until usable data becomes avaialble, we can only speculate as to exactly how.

In closing, to address the titular question of whether or not tags function as retrieval aids, the best answer would appear to be ``sometimes''. While there is much work to be done on when and why particular tags serve this function and others do not, it is clear that the overarching retrieval assumption is far from valid: Tags certainly do not always function as memory cues, and our results suggest that retrieval may actually be among one of the least common of tagging motivations. 

